%
% this file is encoded in utf-8
%

%% 這些設定值將會用於呈現在首頁上,請進行填入

%
% 中文論文設定值,請根據以下的範例進行填入
%

% 論文題目(中文)
\newcommand\cTitle{基於沙盒系統之程式評測應用}

% 我的姓名 (中文)
\newcommand\myCname{黃漢軒}

% 指導教授的姓名 (中文),使用頓號隔開
\newcommand\advisorCname{郭忠義 博士}

% 校名 (中文)
\newcommand\univCname{國立臺北科技大學}

% 系所名 (中文)
\newcommand\deptCname{資訊工程系碩士班}

% 學位名 (中文)
\newcommand\degreeCname{碩士}

% 口試年份 (中文、民國)
\newcommand\cYear{一百一十二}

% 口試月份 (中文)
\newcommand\cMonth{七} 

% 畢業學期(中文)
\newcommand\cGraduateSemester{二}

%
% 英文論文設定值,請根據以下的範例進行填入
%

% 論文題目 (英文)
\newcommand\eTitle{Online Judge System based on Sandbox System}

% 我的姓名 (英文)
\newcommand\myEname{Huang, Han-Xuan}

% 指導教授的姓名 (英文),使用逗號隔開
% 例如:Dr. Kuo Jong-Yi, Dr. A B-C, ...
\newcommand\advisorEname{Dr. Kuo Jong-Yi}

% 校名 (英文)
\newcommand\univEname{National Taipei University of Technology}

% 系所全名 (英文)
\newcommand\fulldeptEname{Department of Computer Science and Information Engineering}

% 系所短名 (英文, 用於書名頁學位名領域)
\newcommand\deptEname{Computer Science and Information Engineering}

% 學院英文名 (如無,則以空的大括號表示)
\newcommand\collEname{}

% 學位名 (英文)
\newcommand\degreeEname{Master of Science}

% 口試年份 (阿拉伯數字、西元)
\newcommand\eYear{2023} 

% 口試月份 (英文)
\newcommand\eMonth{January}

% 學校所在地 (英文)
\newcommand\ePlace{Taipei, Taiwan, Republic of China}

%畢業級別;用於書背列印;若無此需要可忽略
\newcommand\GraduationClass{111}

%%%%%%%%%%%%%%%%%%%%%%