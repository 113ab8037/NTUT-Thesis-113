% 中文摘要頁
\begin{ZhAbstract}
    \begin{ZhAbstractItems}
        % 論文名稱,請在 ntut-labels.tex 定義
        \noindent \text 論文名稱:\cTitle

        % 論文頁數,請自己填
        \noindent \text 頁數:(請自己填)頁

        % 校所別,請在 ntut-labels.tex 定義
        \noindent \text 校所別:\univCname \space \deptCname

        % 畢業時間,請在 ntut-labels.tex 定義
        \noindent \text 畢業時間:\cAcademicYear 學年度 \space 第\cGraduateSemester 學期

        % 學位,請在 ntut-labels.tex 定義
        \noindent \text 學位:\degreeCname

        % 研究生,請在 ntut-labels.tex 定義
        \noindent \text 研究生:\myCname

        % 指導教授,請在 ntut-labels.tex 定義
        \noindent \text 指導教授:\advisorCname

        % 關鍵詞,請自己填,請自己填,多個關鍵字以逗號(、)隔開
        \noindent \text 關鍵詞:(請自己填)

    \end{ZhAbstractItems}

    \begin{ZhAbstractDescription}
        摘要為論文或報告的精簡概要,其目的是透過簡短的敘述使讀者大致瞭解整篇報告的內容。摘要的內容通常須包括問題的描述以及所得到的結果,但以不超過 500 字或一頁為原則,且不得有參考文獻或引用圖表等。以中文撰寫之論文除中文摘要外,得於中文摘要後另附英文摘要。標題使用 20pt 粗標楷體並於上、下方各空一行(1.5 倍行高,字型 12pt 空行)後鍵入摘要內容。摘要頁須編頁碼(小寫羅馬數字表示頁碼)。
    \end{ZhAbstractDescription}
    
\end{ZhAbstract}

